\subsection{Seismic Data Iteration Design}

The primary objective of this research was to gather a comprehensive profile of each seismic attribute's memory consumption.
To achieve this, it was essential to execute the same attribute on different shapes of seismic datasets.

Understanding the nature of seismic datasets is crucial in this context and, typically, seismic data is three-dimensional, encompassing dimensions like inline, crossline, and time or depth slices.
These dimensions represent different aspects of the seismic survey grid, with inline and crossline denoting horizontal measurements and the third dimension representing vertical depth or time.

For the purpose of this research, we adopted a strategic approach to vary the dimensions of the seismic datasets while keeping a consistent structure.
Specifically, the latter two dimensions of the dataset, often corresponding to crossline and time/depth, were fixed at a size of 200.
This decision was made to maintain a level of consistency across different iterations, allowing for a more focused analysis on the impact of varying the first dimension - typically the inline dimension.

The variation in the first dimension was systematically implemented in steps of 200.
This approach enabled to explore a range of dataset sizes, starting from a relatively small dataset and progressively moving towards larger and more complex ones.
By incrementing the size in a controlled manner, it was possible to meticulously observe the changes in memory consumption and understand how scalability and dataset dimensions affect the memory footprint of seismic data processing.

Each iteration of the experiment, therefore, involved executing the same seismic attribute processing task on a dataset whose first dimension varied in size, while the other two dimensions remained constant.
This iterative process provided a detailed and nuanced profile of memory consumption across various dataset sizes, offering valuable insights into the efficiency and scalability of seismic data processing algorithms in high-performance computing environments.
Through this detailed iteration design, the study aimed to establish a clear understanding of how different dataset shapes influence memory usage, which is a critical consideration in optimizing seismic data processing tasks for large-scale geological explorations.

\DF{Aqui talvez seja bom uma imagem simples, mostrando de uma visão geral a questão das iterações}
\DF{Um ponto de dúvida aqui é: não fizemos um experimento formal que mostre que variando a D2 e D3 não muda. Eu rodei isso mas não tem o experimento desenhado. Vale ter isso? No caso pra mostrar que variar a inline da na mesma que variiar crossline our tempo do ponto de vista de memória}