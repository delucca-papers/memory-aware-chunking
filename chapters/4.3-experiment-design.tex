\subsection{Experiment Design}
\label{subsec:experiment-design}

The foundational structure for all experiments in this study follows a standardized and replicable process, designed to ensure consistency and accuracy in the data collection.
This structure is centered around the use of a shell script that initiates a Docker\footurl{https://www.docker.com/} container for each iteration of the experiment.
An iteration here refers to a single execution of the seismic data processing task with a specific input data shape.

This process is divided into four distinct phases:
(i) on the first phase, a shell script initializes the Docker container to execute an iteration for a given data shape and seismic attribute;
(ii) the second phase happens within the Docker container itself, where a synthetic dataset is tailored to the specific iteration based on the expected input data shape;
(iii) the third phase involves the execution of the seismic attribute processing task on the synthetic dataset; and
(iv) the fourth and final phase is getting a summary of the result after all iterations are executed.
\DF{Aqui acho que vale colocar uma imagem ilustrando o processo}

It is important to notice that this containerized approach ensures a controlled and isolated environment for each experiment, minimizing external influences on the memory consumption data.
Also, the analysis of various memory management techniques conducted in the initial stages of this research made it clear that utilizing the /proc filesystem, specifically the smaps\_rollup file, offered a more accurate and detailed understanding of memory consumption.
Based on this, the containerized execution of each iteration is designed to pause and wait for a \textit{CONT} signal\footurl{https://man7.org/linux/man-pages/man7/signal.7.html} on three specific phases: before phase (ii), before phase (iii), and after phase (iii).
The shell script watches the execution of the experiment, and fetches the memory consumption data from the /proc filesystem at each pause, sending the \textit{CONT} signal to the container to continue the execution.
With this approach it is possible to get a detailed memory consumption profile for each iteration, including the memory consumption of the data generation phase and the seismic attribute execution phase.

This structured approach to executing each iteration of the experiment allows for a systematic and detailed collection of memory usage data.
Controlling the execution flow and precisely timing the memory measurements ensures that the data accurately reflects the memory consumption patterns at each stage of the seismic data processing task.
This methodology is critical in enabling a deep understanding of how different input data shapes and processing tasks influence overall memory usage in a high-performance computing environment.

\DF{Em algum lugar preciso colocar um link pro repositório com os experimentos}