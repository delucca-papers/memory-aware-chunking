\subsection{Validation with Real Seismic Data}

The final phase of the research aimed to validate the memory consumption patterns observed in the previous experiments, which utilized synthetic seismic datasets.
The key question to answer was whether the use of real seismic data would result in significantly different memory usage when processing seismic attributes.
This validation process was crucial in ensuring that the findings could be applied to real-world scenarios and geological explorations.

To accomplish this, the research focused on conducting a comparative analysis by executing the same memory consumption experiment on two distinct sets of seismic data: the F3 dataset and the Parihaka dataset.
\DF{Não sei qual a referência pra esses datasets}
These real seismic datasets provided the opportunity to assess memory usage in a more authentic context.

The first step involved selecting seismic attributes that were commonly used in real-world seismic data processing tasks.
It was also important to ensure that these attributes were representative of the types of computations typically performed in geological explorations.

Next, it was necessary to generate synthetic datasets that precisely matched the dimensions of the real F3 and Parihaka datasets.
These synthetic datasets served as controlled counterparts to the real data, allowing to isolate the influence of data source (real vs. synthetic) on memory consumption.

Subsequently, the execution followed the same same process outlined in section \ref{subsec:experiment-design}, executing the chosen seismic attributes on both the real and synthetic datasets.
This included the careful timing of memory measurements at various phases of the experiment, as well as the collection of detailed memory consumption profiles.

By comparing the memory usage patterns between the real seismic data and their synthetic counterparts, it would be possible to identify any significant discrepancies.
This analysis provided valuable insights into how the origin of the data source impacts memory consumption during seismic attribute processing.

\DF{Esse aqui a gente começou a executar e já viu as diferenças, então precisa formalizar e formatar}