\subsection{Memory Reporting Techniques}

This investigation into memory reporting techniques evaluates various tools for accurate memory usage gauging across different scenarios.
Each technique employs a distinct strategy for reporting memory usage, highlighting the variability and complexity in memory profiling.
The findings summary presented in Figure \ref{fig:memory-reporting-summary} elucidates the strengths and weaknesses of each method under different conditions.

\begin{figure}[ht]
    \centering
    \includegraphics[width=\linewidth]{artifacts/result-memory-reporting-techniques.jpg}
    \caption{Summary of memory reporting techniques}
    \label{fig:memory-reporting-summary}
\end{figure}

Analysis reveals that applications with minimal memory usage demonstrate clear disparities between the outcomes of different reporting techniques.
However, as memory usage scales, these differences become significantly less pronounced.
For larger-scale applications, the choice of memory reporting tool might minimally impact the perceived efficiency and effectiveness of memory usage profiling.

The study finds that the \texttt{mprof} tool accurately measures peak memory usage but falls short in evaluating the final memory footprint of an application.
Additionally, the overhead introduced by \texttt{mprof} due to profiling can lead to noticeable application performance slowdowns, potentially skewing memory usage analysis results.

Conversely, \texttt{psutil} fails to measure peak memory usage, offering only snapshots of memory consumption at specific instances.
This limitation renders it less suitable for applications where peak memory usage is critical.
Similarly, the \texttt{resource} module can report initial and peak memory usage but cannot track the final memory footprint, vital for comprehensive memory usage analysis.

Given these limitations, direct memory usage data querying from the Linux kernel using \texttt{/proc/pid/status} emerged as a more reliable method.
This approach accurately predicts initial, final, and peak memory usages without the drawbacks observed in other techniques.
Direct querying provides a holistic view of an application's memory footprint and circumvents the performance penalties associated with profiling tools.

In conclusion, while no single memory reporting technique emerges as universally superior, the findings advocate for a hybrid approach leveraging the strengths of each method.
For comprehensive memory profiling, direct kernel querying serves as a robust foundation, supplemented by targeted use of tools like \texttt{mprof}, \texttt{psutil}, and \texttt{resource} for specific scenarios.
