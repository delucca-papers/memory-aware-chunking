\subsection{Memory Management Techniques}

\DF{Aqui precisamos construir um experimento formal que compare e mostre as diferenças em cada ferramenta}

The first phase of the research centered around establishing a reliable framework for measuring memory consumption.
Evaluating various tools and techniques was necessary to achieve this, including: mprof\footurl{https://github.com/pythonprofilers/memory\_profiler}, psutil\footurl{https://github.com/giampaolo/psutil}, and resource\footurl{https://docs.python.org/3/library/resource.html} modules.
These tools are particularly suited for Python environments, offering insights into dynamic memory allocation and real-time memory utilization.
However, considering the high-level nature of Python and its abstraction from direct memory management, it was necessary to supplement these tools with direct measurements from the Linux /proc\footurl{https://man7.org/linux/man-pages/man5/proc.5.html} filesystem, specifically using the smaps interface.
This approach allowed to obtain a more granular understanding of memory usage, which is essential when dealing with large datasets typically encountered in seismic data processing.