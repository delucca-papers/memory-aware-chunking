\chapter{Introduction}
\label{ch:introduction}

\DF{Vale a pena dividir em seções? Se sim, eu penso em ser: Motivação e Contexto, Background, Limitações do Auto-Chunking do Dask, Motivação para o Memory-Aware Chunking, Minha Contribuição, e Estrutura da Tese}

\DFTODO{Começar com uma base da motivação e contexto}

\DFTODO{Na motivação e contexto, começar pela motivação geral, explicando o problema da gestão de memória para o processamento em larga escala de dados, focando em sistemas distribuídos}

\DFTODO{Explicar sobre o quão comum é esse tipo de aplicação usar o Dask para paralelizar cargas de trabalho}

\DFTODO{Explicar sobre os desafios para o gerenciamento de recursos. Aprofundar a motivação para explicar que alguns casos geram um grande consumo de memória e que, nesses casos, é necessário estimar o consumo de memória para definir o tamanho do chunk}

\DFTODO{Trazer o probleam base, dos problemas que um chunk definido de maneira não adequada pode trazer: o excesso de tentativa e erro, o uso subótimo dos recursos, o overhead do escalonador}

\DFTODO{Trazer a motivação para o memory-aware chunking, explicando que é necessário considerar o consumo de memória para definir o tamanho do chunk ideal. Nesse ponto, citar que esse problema é particularmente relevante para algoritmos que consomem muita memória}

\DFTODO{Trazer a motivação para a contribuição, explicando que o objetivo é criar um método para estimar o consumo de memória com poucas execuções. Esse método remove a necessidade de tentativa e erro e cria uma forma prática de definir o tamanho de chunk ideal}

\DFTODO{Trazer a estrutura da tese, explicando como os capítulos estão organizados}