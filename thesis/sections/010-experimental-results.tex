\chapter{Experimental Results}
\label{ch:results}

\DF{Este capítulo pode ser dividido em seções para melhor organização. Eu considero incluir: Comparação de Ferramentas de Medição de Memória, Impacto de Execuções com Memória Limitada, Análise do Comportamento de Memória dos Algoritmos, Avaliação do Desempenho do Dask com Diferentes Configurações de Chunk, Comparação da Auto-Fragmentação do Dask, Consumo de Memória por Algoritmo no Dask, Desenvolvimento de um Previsor de Consumo de Memória, Definição do Melhor Tamanho de Chunk e Previsão com Base no Formato dos Dados de Entrada.}


\section{Comparison of Memory Measurement Tools}
\label{sec:memory_tools}

\DFTODO{Descrever o primeiro experimento realizado para comparar a coleta do consumo de memória usando diferentes ferramentas.}

\DFTODO{Detalhar os desafios enfrentados devido à sensibilidade dos experimentos e como pequenos descuidos podiam introduzir vieses nos resultados.}


\section{Impact of Executions with Limited Memory}
\label{sec:limited_memory}

\DFTODO{Avaliar os resultados de executar os algoritmos com menos memória do que o medido, observando que eles funcionam, mas com impacto no tempo de execução.}

\DFTODO{Discutir as implicações desse comportamento para o dimensionamento de recursos e desempenho.}


\section{Analysis of Algorithms’ Memory Behavior}
\label{sec:algorithm_memory_behavior}

\DFTODO{Apresentar os experimentos que focaram em avaliar o comportamento do consumo de memória dos algoritmos.}

\DFTODO{Explicar como, trabalhando com dados 3D, foram feitos testes mantendo uma dimensão fixa, duas e sem nenhuma, para avaliar o resultado do comportamento de memória.}


\section{Evaluation of Dask’s Performance with Different Chunk Configurations}
\label{sec:dask_chunk_performance}

\DFTODO{Descrever os experimentos realizados com o Dask, avaliando como ele se comporta dividindo o mesmo dado em diferentes formatos de chunk e comparando o tempo de execução.}

\DFTODO{Analisar os resultados e discutir como o tamanho e a forma dos chunks afetam o desempenho.}


\section{Comparison of Dask’s Auto-Chunking}
\label{sec:dask_auto_chunking}

\DFTODO{Apresentar os experimentos feitos para avaliar e comparar como o auto-chunking do Dask funciona, mostrando que ele ignora o consumo de memória do grafo.}

\DFTODO{Discutir as limitações identificadas e as implicações para algoritmos que consomem muita memória.}


\section{Memory Consumption per Algorithm in Dask}
\label{sec:memory_per_algorithm}

\DFTODO{Avaliar o consumo de memória por algoritmo usando o Dask, detalhando as diferenças observadas entre eles.}

\DFTODO{Comparar os algoritmos Envelope, GST3D e Filtro Gaussiano 3D em termos de consumo de memória e tempo de execução.}


\section{Development of a Memory Consumption Predictor}
\label{sec:memory_predictor}

\DFTODO{Descrever o processo de criação de um preditor do consumo de memória com base nas execuções do Dask.}

\DFTODO{Explicar a metodologia utilizada para prever o consumo de memória com base em algumas execuções do mesmo algoritmo contra diferentes entradas.}


\section{Definition of the Best Chunk Size}
\label{sec:best_chunk_size}

\DFTODO{Utilizando o preditor desenvolvido, apresentar como foi definida uma estrutura para determinar o melhor tamanho de chunk.}

\DFTODO{Discutir a necessidade de considerar o paralelismo nessa definição e como isso foi abordado.}


\section{Prediction Based on Input Data Format}
\label{sec:input_data_prediction}

\DFTODO{Descrever como, por fim, foi montada uma estrutura capaz de prever o consumo de memória apenas com o formato do dado de entrada.}

\DFTODO{Discutir as vantagens dessa abordagem e suas possíveis aplicações em diferentes cenários.}